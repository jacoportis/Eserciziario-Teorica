\documentclass[12pt]{book}%openany toglie la pagine bianche tra i capitoli

% Language setting
% Replace `english' with e.g. `spanish' to change the document language
\usepackage[italian]{babel}
\newcommand{\E}{È \hspace{0.1mm}}
\usepackage{import}
% Set page size and margins
% Replace `letterpaper' with `a4paper' for UK/EU standard size
\usepackage[letterpaper,top=2cm,bottom=2cm,left=3cm,right=3cm,marginparwidth=1.75cm]{geometry}

\usepackage{amssymb} % for harpoons
% Useful packages
\usepackage[utf8]{inputenc}
\usepackage{afterpage}
\newcommand\blankpage{%
    \null
    \thispagestyle{empty}%
    \newpage}
\usepackage{amsmath}
\usepackage{graphicx}
\usepackage[psdextra, colorlinks=true, allcolors=black]{hyperref}
\usepackage{array}% only needed for injecting commands at the beginning of columns in the tabular below
\usepackage{ambienti} %per gli esercizi
\usepackage[version=4]{mhchem}

\newcommand\myline[1][]{%
  \,\tikz[baseline]\draw[very thick,#1](0,-\dp\strutbox)--(0,\ht\strutbox);\,%
}
\usepackage{enumitem}
\usepackage{float}
\usepackage{mathtools}
\usepackage{tikz}
\usetikzlibrary {shapes.geometric}
\usetikzlibrary{decorations.markings}
\usetikzlibrary{positioning}
\usepackage{lipsum}
\usepackage{chemfig}
\usepackage{amsmath}
\usepackage{array}
\setlength\parindent{0pt}%e si gode, toglie lo spostamento a destra di una nuova riga
\usepackage{textgreek}
\usepackage[open]{bookmark}
\ProvidesFile{puenc-greek.def}
%\input{puenc-greekbasic.def}
\usepackage{caption}
\usepackage{float}

\makeatletter
\newcommand\mathcircled[1]{%
  \mathpalette\@mathcircled{#1}%
}
\newcommand\@mathcircled[2]{%
  \tikz[baseline=(math.base)] \node[draw,ellipse,inner sep=1pt] (math) {$\m@th#1#2$};%
}
\makeatother

%impaginazione
\usepackage{fancyhdr}

\begin{document}

\thispagestyle{empty}
\begin{center}

\begin{minipage}[c]{0.45\textwidth}
\begin{flushleft}
\includegraphics[width=0.8\textwidth]{logo-unict-orizzontale-grigio.png}
\end{flushleft}
\end{minipage}
\hfill
\begin{minipage}[c]{0.45\textwidth}
\begin{flushright}
\includegraphics[width=\textwidth]{logo_dfa_orizzontale}
\end{flushright}
\end{minipage}\\
\medskip
\hbox to \textwidth{\hrulefill}

\vfill
\vfill

\uppercase{\sc{ \Large{\textbf{Esercizi svolti di Fisica Teorica}}}}\\

\vfill
\large{A cura di P. Salumieri}

\vfill
\vfill
\hbox to \textwidth{\hrulefill}
{\sc anno 2025}
\end{center}

\afterpage{\blankpage}
\newpage

\thispagestyle{empty}


\afterpage{\blankpage}

%\frontmatter

\clearpage                       % Otherwise \pagestyle affects the previous page.
{                                % Enclosed in braces so that re-definition is temporary.
  \pagestyle{empty}              % Removes numbers from middle pages.
  \fancypagestyle{plain}         % Re-definition removes numbers from first page.
  {
    \fancyhf{}%                       % Clear all header and footer fields.
    \renewcommand{\headrulewidth}{0pt}% Clear rules (remove these two lines if not desired).
    \renewcommand{\footrulewidth}{0pt}%
  }
  \tableofcontents
  \thispagestyle{empty}          % Removes numbers from last page.
} %roba per mettere l'indice senza numero di pagina ne marks

\newpage

\pagestyle{fancy}
\fancyhf{}
\fancyhead[LE]{\nouppercase{\textbf{\thepage}\hfill\leftmark}}
\fancyhead[RO]{\nouppercase{\rightmark\hfill\textbf{\thepage}}}
\fancypagestyle{plain}{%
\fancyhf{} % cancella tutti i campi di intestazione e pi\‘e di pagina
%\fancyfoot[C]{\bfseries \thepage} % tranne il centro
\renewcommand{\headrulewidth}{0pt}
}

%\tableofcontents

\thispagestyle{empty}

\chapter{Esercizi di Fisica Teorica}
\chapter{Eserciziiiiboooh}

\begin{esercizio}[pippo]
   ciao
\end{esercizio}

\end{document}