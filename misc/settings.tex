%=================%
%   PAGE LAYOUT   %
%=================%

\settypeblocksize{*}{1.3\lxvchars}{*}   % lxvchars è una dimensione raccomandata che dipende dal font, utile per la leggibilità
\setlrmargins{*}{*}{1}                  % setta i margini in modo che siano 1:1 adeguandosi al typeblock
\setulmarginsandblock{1.5in}{*}{1}    % setta i margini superiore e inferiore {superiore}{inferiore}{rapporto}
\checkandfixthelayout                   % does the magic



%=========================%
%   HEADER & PAGESTYLES   %
%=========================%

\setlength{\headwidth}{\textwidth}

\makepagestyle{thesis}                                                     % definisce il nome dello stile
\makerunningwidth{thesis}{\headwidth}                                      % definisce la lunghezza dell'header
\makeheadrule{thesis}{\headwidth}{\normalrulethickness}                    % mette la riga nell'header
\makeheadposition{thesis}{flushright}{flushleft}{flushright}{flushleft}    % definisce la posizione dell'header
\makepsmarks{thesis}{%                                                     % da qui in poi non lo so
  \nouppercaseheads
  \createmark{chapter}{both}{shownumber}{\chaptername\ }{.\ } 
  \createmark{section}{right}{shownumber}{}{.\ }              
  \createplainmark{toc}{both}{\contentsname}                  
  \createplainmark{lof}{both}{\listfigurename}                
  \createplainmark{lot}{both}{\listtablename}
  \createplainmark{bib}{both}{\bibname}
  \createplainmark{index}{both}{\indexname}
  \createplainmark{glossary}{both}{\glossaryname}    
}

%% se oneside:
  % \makeoddhead{thesis}{\bfseries\leftmark}{}{\bfseries\thepage}
%% se twoside:
  \makeevenhead{thesis}{\bfseries\thepage}{}{\bfseries\leftmark}
  \makeoddhead{thesis}{\bfseries\rightmark}{}{\bfseries}

\aliaspagestyle{chapter}{empty}
\chapterstyle{hangnum}


%===============%
%   NUMBERING   %
%===============%

\numberwithin{equation}{chapter}    % Aggiunge il numero del capitolo all'equazione
\setsecnumdepth{subsection}         % numera le subsection (1.1.1)

\hypersetup{                    % ======================================
    colorlinks=true,            % template preso da internet
    linkcolor=black,            % molto sobrio, le cose cliccabili si evidenziano quando passi il mouse
    filecolor=black,            %
    urlcolor=blue,              %
    pdfpagemode=FullScreen,     %
    citecolor=black,            %
}



\setlength\parindent{0pt}%e si gode, toglie lo spostamento a destra di una nuova riga
%
%   TIKZ
%
\usetikzlibrary{shapes.geometric}
\usetikzlibrary{decorations.markings}
\usetikzlibrary{positioning}



%
%   ALTRO / COMANDI / MISCELLANEA
%

\newcommand\myline[1][]{%
  \,\tikz[baseline]\draw[very thick,#1](0,-\dp\strutbox)--(0,\ht\strutbox);\,%
}

\newcommand{\E}{È}

\makeatletter
\newcommand\mathcircled[1]{%
  \mathpalette\@mathcircled{#1}%
}
\newcommand\@mathcircled[2]{%
  \tikz[baseline=(math.base)] \node[draw,ellipse,inner sep=1pt] (math) {$\m@th#1#2$};%
}
\makeatother